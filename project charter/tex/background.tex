In the rapidly evolving landscape of warehousing and logistics, the conventional approach to material handling in warehouses contains several challenges. Palletizing and depalletizing boxes manually is a time-consuming task, which decreases productivity. Consequently, labor costs associated with manual palletizing and depalletizing are increased, encompassing wages, benefits, and additional expenditures associated with managing a workforce engaged in repetitive physical activities. Moreover, manual material handling poses inherent risks to the workforce. Accidents and injuries are a constant concern, impacting both employee well-being and company reputation. The physically demanding nature of tasks can also lead to reduced job satisfaction. Finally, with modern commerce continuing to grow, warehouses must adapt to higher volumes of goods and fulfill needs with efficiency. Manual processes often hinder the ability to meet customer demands effectively.

The integration of robotic arms for palletizing and depalletizing tasks is no longer a matter of convenience, but a necessity. The shift towards automation is driven by the pitfalls of manual labor listed above, as well as several compelling factors consistent in robotic arms that demonstrate its ability to address these challenges effectively and revolutionize material handling in warehouses. Robotic arms excel at repetitive tasks requiring precision. The consistency of robot arms translates to significant improvements in efficiency, ultimately allowing faster order fulfillment and throughput rates. Robotic arms are also largely scalable. With a large rise in e-commerce, robotic arms can adapt to seasonal demand, and can shift workload to accommodate responsively in a dynamic marketplace. In addition to their agility in a demanding environment, robotic arms provide product integrity. In industries like food and pharmaceuticals, robotic arms are programmed to handle goods aligning with quality standards issued by the Occupational Safety and Health Administration (OSHA). This not only guarantees product integrity, but also compliance with industry-specific regulations.

The University of Texas at Arlington has a Mitsubishi Electric RV-8CRL Robot housed in room 335 of the Engineering Research Building. Previous teams of students and Dr. Christopher McMurrough have worked on the hardware setup of the work cell, including the MELSEC Programmable Logic Controller (PLC). The RV-8 also is mounted to a linear rail, enabling its 7th axis. While the robot is operational, the work cell still needs several areas of integration completed. The laser safety scanners create virtual safety zones around the robot, monitoring the surrounding area for any human intrusion. These laser scanners are not implemented yet. There are two emergency stop (E-stop) switches in and around the work cell, providing immediate and highly visible means to shut down the robot in the case of an emergency. Currently, the emergency stops are functional but wired inefficiently. The linear rail was most recently installed with aims to enhance reach and flexibility of the robot. At present, the linear rail is not integrated with the RV-8 robot and is not functional.

The sponsor of this project, Dr. McMurrough, aims for this project to finish the integration of the work cell, with the hopes that it serves as an invaluable resource for both research and educational purposes at UTA. The robot is within a glass encasing shown to onlookers in the building. Thus, this project hopes to showcase the functionalities of the robot to a wider audience and spark interest at the consistent innovation within Computer Science and Engineering at UTA. The completed work cell will empower students with hands-on experience, allowing them to explore the frontiers of robotics, automation, and control systems. 
