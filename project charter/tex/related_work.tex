
Factory automation, including robotic painting systems, is emerging as a state-of-the-art technology in manufacturing. There are several current implementations of robotics that address the current challenges in factory automation. Several companies offer robotic systems designed specifically for painting applications. A pioneer in these types of robots is ABB's Delta line of robots, most notably the FlexPicker IRB 360 \cite{robotsdonerightHistoryRobots}. The first prototype was designed by a research group led by Professor Reymond Clavel at EPFL, Switzerland \cite{bonev_delta_2001}. The groundbreaking robot was based on the concept of parallelograms, where three parallelograms restricted the mobile platform's orientation to three purely translational degrees of freedom. The robot's actuation involves rotating levers with revolute joints, which can be actuated using rotational motors or linear actuators. Its lightweight construction and base-mounted actuators make it ideal for light objects between 10 grams to 1 kilogram \cite{bonev_delta_2001}. Covered by over 30 patents, the robot's success is seen in several industries including automotive manufacturing, electronics, and aerospace.

Another notable robotic system in factory automation from ABB's product line is the IRB 6700 Series. Similar to the RV-8CRL, an IRB 6700 robot is a vertical robotic solution with six degrees of freedom. The series has an impressive payload range between 150 to 300 kilograms, with a max reach of 3.2 meters \cite{robotics2021product}. The line even includes variants available as floor mounted and inverted versions, matching the orientation of UTA's RV-8CRL. Other series designed by ABB include the IRB 2600 line, IRB 140 line, and YuMi robots.

FANUC's M-20 series of robots represent a versatile solution for medium payload painting tasks in manufacturing. This line of robots have a payload capacity of 35 kilograms and a max reach of 1,811 millimeters \cite{connolly2007new}. The robot is strong, yet reasonably light. The 6-axis machine's design incorporates a hollow upper arm, and is ideal for multi-material handling. Its compact footprint allows it to work efficiently in confined spaces, optimizing space utilization within the factory. FANUC's dedication to reliability is reflected in the M-20 series, as it boasts an industry leader long Mean Time Between Failures (MTBF) figure, up to 11 years \cite{aggogeri2020robotic}. Other series designed by FANUC include the CR series and R-30iB series.

KUKA, a German robotics company, offers a range of robots suited for painting and automation tasks in manufacturing. Their KR AGILUS series, for example, provides high-speed and precise handling capabilities for small and medium payloads \cite{muftooh20186}. On the other hand, Yaskawa Motoman, a subsidiary of Yaskawa Electric Corporation, specializes in industrial robots for various applications, including painting in manufacturing. The Motoman MH series offers versatility and efficiency in painting tasks \cite{muftooh20186}. Additionally, companies like Universal Robots and MiR (Mobile Industrial Robots) have gained prominence for their collaborative and autonomous mobile robots that can work alongside human workers in factory environments.

As robots are becoming more technologically available, manufacturing industries are adopting the technology with frequency. According to ABI Research, a global tech market advisory firm, worldwide commercial robot revenue in factories will have a Compounded Annual Growth Rate (CAGR) of over 23% from 2021 to 2030 and exceed $51 billion by 2030 \cite{nytimesRobotsArent}. This is seen with industry leaders like automotive manufacturers, who are increasingly investing in robotic painting systems within their production facilities.

Locally, companies are building factories with an aim at reducing human labor. Companies like Tesla have been researching with their robotic painting systems, which have capabilities of painting and coating various automotive parts. Operating at a facility in Dallas, Texas, these systems are capable of handling a significant portion of the painting process \cite{dallasnewsBattleHumans}. As these companies continue to innovate and invest in automation solutions, the manufacturing industry is poised for a transformative shift marked by increased efficiency, reduced operational costs, and improved product quality.