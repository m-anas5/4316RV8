The RV8 Robot Work Cell project will be situated in ERB 335, a designated laboratory area. This lab area will include the electrical infrastructure to accommodate the power requirements of the RV8 robot, MELSEC, and PLC controllers. Safety measures are of paramount importance, and as such, emergency stop systems, safety barriers, and warning signs have been implemented and will be modified to create a secure working environment for both operators and equipment. Additionally, the layout of workstations within ERB 335 will be configured to optimize workflow efficiency, allowing operators to interact with the RV8 robot and associated equipment.

In terms of equipment, the RV8 Robot Work Cell will feature essential components. The RV8 robot itself will be the central robotic arm, equipped with advanced technology to provide precision, speed, and versatility in performing a wide range of tasks. The project will also incorporate a Mitsubishi Electric MELSEC controller as the central control unit, facilitating  coordination and synchronization of the robot's actions. Programmable Logic Controllers (PLCs) will be  positioned to manage auxiliary equipment and processes within the work cell Furthermore, to complement the RV8 robot's capabilities, a gripper will be outsourced from a reputable supplier, selected based on its compatibility and suitability for handling various materials and objects within the work cell. Additionally, the work cell will incorporate a linear rail with a motor attached to it, a key feature for controlled and precise linear movement of the RV8 robot. This motor will be  connected to the PLC controller, allowing for precise control over the linear motion of the robot along the rail. The linear rail will be provided in the laboratory as well and will require reading of several manuals from Mitsubishi to configure the movement of the linear rail.
