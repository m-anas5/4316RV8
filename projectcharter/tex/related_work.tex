Warehouse robotics is emerging as a state-of-the-art technology. There are several current implementations of robotics that address the current challenges in logistics. Several companies offer robotic systems designed specifically for picking and placing (or picking and packaging). A pioneer in these types of robots is ABB's Delta line of robots, most notably the FlexPicker IRB 360 \cite{robotsdonerightHistoryRobots}. The first prototype was designed by a research group led by Professor Reymond Clavel at EPFL, Switzerland \cite{bonev_delta_2001}. The groundbreaking robot was based on the concept of parallelograms, where three parallelograms restricted the mobile platform's orientation to three purely translational degrees of freedom. The robot's actuation involves rotating levers with revolute joints, which can be actuated using rotational motors or linear actuators. Its lightweight construction and base-mounted actuators make it ideal for light objects between 10 grams to 1 kilogram \cite{bonev_delta_2001} Covered by over 30 patents, the robot's success is seen in several industries including food/beverage packaging, electronics, and pharmaceuticals. 

Another notable robotic system in warehouse automation from ABB's product line is the IRB 6700 Series.  Similar to the RV-8CRL, an IRB 6700 robot is a vertical robotic solution with six degrees of freedom. The series has an impressive payload range between 150 to 300 kilograms, with a max reach of 3.2 meters \cite{robotics2021product}. The line even includes variants available as floor mounted and inverted versions, matching the orientation of UTA's RV-8CRL. Other series designed by ABB include the IRB 2600 line, IRB 140 line, and YuMi robots.

FANUC's M-20 series of robots represent a versatile solution for medium payload material handling tasks in warehousing and manufacturing. This line of robots have a payload capacity of 35 kilograms and a max reach of 1,811 millimeters \cite{connolly2007new}. The robot is strong, yet reasonably light. The 6-axis machine's design incorporates a hollow upper arm, and is ideal for multi-material handling. Its compact footprint allows it to work efficiently in confined spaces, optimizing space utilization within the warehouse. FANUC's dedication to reliability is reflected in the M-20 series, as it boasts an industry leader long Mean Time Between Failures (MTBF) figure, up to 11 years \cite{aggogeri2020robotic}. Other series designed by FANUC include the CR series and R-30iB series.

KUKA, a German robotics company, offers a range of robots suited for material handling and automation tasks in logistics. Their KR AGILUS series, for example, provides high-speed and precise handling capabilities for small and medium payloads \cite{muftooh20186}. On the other hand, Yaskawa Motoman, a subsidiary of Yaskawa Electric Corporation, specializes in industrial robots for various applications, including material handling in warehouses. The Motoman MH series offers versatility and efficiency in material handling tasks \cite{muftooh20186}. Additionally, companies like Universal Robots and MiR (Mobile Industrial Robots) have gained prominence for their collaborative and autonomous mobile robots that can work alongside human workers in warehouse environments. 

As robots are becoming more technologically available, warehouses and logistics industries are adopting the technology with frequency. According to ABI Research, a global tech market advisory firm, worldwide commercial robot revenue in warehouses will have a Compounded Annual Growth Rate (CAGR) of over 23\% from 2021 to 2030 and exceed \$51 billion by 2030 \cite{nytimesRobotsArent}. This is seen with industry leaders Amazon, who in 2021 alone amassed 38 percent of total US warehouse automation investment within their fulfillment centers \cite{interactanalysisWillAmazons}. This comes after their \$1 billion dollar commitment to Industrial Innovation Fund to support robotics firms, and the purchase of Kiva \cite{interactanalysisWillAmazons}.

Locally, companies are building warehouses with an aim at reducing human labor. Amazon has been researching with their Sparrow robot, which has capabilities of picking up and sorting unpackaged items. Operating at a facility in Dallas, Texas, Sparrow is capable of picking up and sorting roughly 65\% of inventory \cite{dallasnewsBattleHumans}. Walmart, FedEx, and Tesla are also prototyping several picking-and-sorting robots in their respective domains \cite{dallasnewsBattleHumans}. As these companies continue to innovate and invest in automation solutions, the warehousing and logistics industries are poised for a transformative shift marked by increased efficiency, reduced operational costs, and improved service delivery.