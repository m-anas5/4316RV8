This project will palletize and depalletize the boxes according to the requirement of the customer defined purpose. The boxes which are less than 7kg in weight will arrange and rearrange the boxes with defined steps as per the operational capacity. This will also implement an extra part to operate functionally for the boxes to keep in required position, QR reader will be implemented. For the safety, the light tower will be implemented to show the operation using different colors.  
\subsection{Barcode reader}
\subsubsection{Description}
By incorporating a QR code or barcode scanner into the RV 8 robot arm, it can be programmed to move to various positions based on the barcode it reads. It can figure out what it is looking at and move accordingly. This means it can neatly stack packages from the ground onto specific rails or take packages off the rail carefully. This scanner helps the robot work better, making it quicker and less likely to make mistakes. It is perfect for jobs where things need to be sorted or moved just right. The RV 8 robot arm can recognize different barcodes, making it really useful for businesses that handle all kinds of stuff.
\subsubsection{Source}
CSE Senior Design project specifications.
\subsubsection{Constraints}
Exceeding weight limitations can lead to performance issues and safety concerns. The budget of the business may limit what can be implemented. The working environment can affect the scanner's performance. Factors like dust, temperature, and humidity might limit its effectiveness and require additional protective measures.
\subsubsection{Standards}
\begin{itemize}
\item ISO/IEC 18004:20xx Information technology - Automatic identification and data capture techniques: The QR code scanner should follow ISO/IEC 18004 for QR code data encoding and data matrix symbols for compatibility and readability.
\item ISO 10218 Robots and robotic devices - Safety requirements for industrial robots: In order to ensure the safe interaction of the robot arm with the QR scanner and other equipment.
\end{itemize}
\subsubsection{Priority}
\begin{itemize}
\item Critical:
The robot's ability to determine its actions and destinations based on QR codes is important. Without this feature, the product would be rendered ineffective.
\end{itemize}
\subsection{Runtime setup}
\subsubsection{Description}
This feature is related to the ability to configure and adjust the time parameters of automated operations or processes within the system. Users can specify the duration of various tasks, ensuring precise control over timing for optimal system performance. This setting allows for the fine-tuning of process duration to meet specific operational needs.
\subsubsection{Source}
Hyun Ho Kim has expressed the need for flexible and customizable runtime settings to adapt the system's performance to their unique requirements. 
\subsubsection{Constraints}
Sustainability: Any changes made to runtime settings should not compromise the system's overall sustainability and energy efficiency.
\subsubsection{Priority}
\begin{itemize}
\item Low:
While the ability to configure runtime parameters is a valuable feature, it is not considered critical to the core functionality of the product at this time.
\end{itemize}
\subsection{Green Light Signal}
\subsubsection{Description}
The green light tower is a visual indicator designed to signal the operational status of a critical system or process. This tower consists of multiple stacked light modules that emit a bright green light when the system is in an operational or "go" status. When the system is in a non-operational or "stop" status, the green light tower remains off. The green light tower serves as a quick and easily visible means of providing real-time status information, enhancing safety and productivity in industrial and manufacturing environments.
\subsubsection{Source}
CSE Senior Design project specifications.
\subsubsection{Constraints}
Health and Safety: It should meet safety standards to prevent potential hazards or misinterpretation of status.
Environmental: The components of the green light tower should be environmentally friendly and compliant with relevant regulations regarding materials and energy consumption.
\subsubsection{Standards}
\begin{itemize}
\item EN ISO 13850: Safety of machinery - Emergency stop function - Principles for design.
\item IEC 60947-5-2:2019 Low-voltage switchgear and controlgear - Part 5-2: Control circuit devices and switching elements - Proximity switches
\end{itemize}
\subsubsection{Priority}
\begin{itemize}
\item High:
This feature is very important for the product as it directly impacts the safety and efficiency of industrial operations. The green light tower is essential for providing real-time status information, enhancing worker safety, and ensuring smooth and error-free processes. Its absence would compromise both operational safety and productivity, making it a fundamental requirement.
\end{itemize}
\subsection{Orange Light Signal}
\subsubsection{Description}
The orange light is a distinctive visual indicator designed to signal specific conditions or events within the system. This light module emits a bright orange light when a predefined condition is met or when a warning or alert status is triggered. It serves as a clear and easily recognizable means of providing immediate visual cues to operators or users, enhancing safety and ensuring efficient response to critical events.
\subsubsection{Source}
CSE Senior Design project specifications.
\subsubsection{Constraints}
Health and Safety: It should meet safety standards to prevent potential hazards or misinterpretation of status.
Environmental: The components of the orange light tower should be environmentally friendly and compliant with relevant regulations regarding materials and energy consumption.
\subsubsection{Standards}
\begin{itemize}
\item EN ISO 13850: Safety of machinery - Emergency stop function - Principles for design.
\item IEC 60947-5-2:2019 Low-voltage switchgear and controlgear - Part 5-2: Control circuit devices and switching elements - Proximity switches
\end{itemize}
\subsubsection{Priority}
\begin{itemize}
\item High:
This feature is very important for the product as it directly impacts the safety and efficiency of industrial operations. The orange light tower is essential for providing real-time status information, enhancing worker safety, and ensuring smooth and error-free processes. Its absence would compromise both operational safety and productivity, making it a fundamental requirement.
\end{itemize}
\subsection{Red Light Signal}
\subsubsection{Description}
The red light signal is a vital visual indicator designed to signal immediate stop or emergency conditions within the system. This light module emits a bright red light when predefined critical events or emergency states occur. It serves as a clear and universally recognized means of conveying urgent visual cues to operators or users, ensuring safety and facilitating quick, decisive responses in emergency situations.
\subsubsection{Source}
CSE Senior Design project specifications.
\subsubsection{Constraints}
Health and Safety: It should meet safety standards to prevent potential hazards or misinterpretation of status.
Environmental: The components of the red light tower should be environmentally friendly and compliant with relevant regulations regarding materials and energy consumption.
\subsubsection{Standards}
\begin{itemize}
\item EN ISO 13850: Safety of machinery - Emergency stop function - Principles for design.
\item IEC 60947-5-2:2019 Low-voltage switchgear and controlgear - Part 5-2: Control circuit devices and switching elements - Proximity switches
\end{itemize}
\subsubsection{Priority}
\begin{itemize}
\item High:
This feature is very important for the product as it directly impacts the safety and efficiency of industrial operations. The red light tower is essential for providing real-time status information, enhancing worker safety, and ensuring smooth and error-free processes. Its absence would compromise both operational safety and productivity, making it a fundamental requirement.
\end{itemize}

\subsection{Palletizing and depalletizing}
\subsubsection{Description}
The main objective of this system is to efficiently arrange and rearrange boxes in a structured manner, sticking to a method specified by the customer. This will be based on the customer's specified method. 
\subsubsection{Source}
Team ROBO CREW 
\subsubsection{Constraints}
In the system design, certain constraints must be considered for functionality. These constraints primarily involve establishing specific offset values that must be configured to align with the desired operational parameters. Before achieving full operational capacity, testing procedures will be necessary to verify and fine-tune these set offset values. These tests will ensure that the system operates within the defined constraints, maintaining accuracy and efficiency in the palletizing and depalletizing processes. 
\subsubsection{Standards}
Not applicable. 
\subsubsection{Priority}
High

\subsection{Installation of Linear Rail}
\subsubsection{Description}
In addition to the robot arm, an important component is required for desire functionality, namely the linear rail. Serving as the system's 7th axis complementing the robot arm's existing 6 axes, the linear rail is a crucial addition. Its primary role is to facilitate backward and forward movements, significantly enhancing the system's versatility. The integration of this linear rail component adds seamless and efficient operations, allowing the robot arm to navigate and position itself with great accuracy and efficiency, meeting the specific customer demands.
\subsubsection{Source}
Team ROBO CREW 
\subsubsection{Constraints}
Linear rail needs to setup properly. 
\subsubsection{Standards}
Not applicable. 
\subsubsection{Priority}
High