In an industrial setting, the performance and efficiency of machinery are pivotal in determining earnings. Any delay, inefficiency, or downtime in industrial operations can have a direct impact on the profitability of an organization. The RV8-CRL has specific performance requirements to adhere to.

\subsection{Load Capacity}
\subsubsection{Description}
The RV8-CRL has a rated payload of 7 kg (15.432 lbs) and a maximum payload of 8 kg (17.637 lbs).
\subsubsection{Source}
This requirement is sourced by Mitsubishi.
\subsubsection{Constraints}
The gripper used in conjunction with the robot must be designed and configured to support the same payload capacity as specified by the robot's rated and maximum payload. Failure to meet this constraint can lead to suboptimal performance and potential damage to the robot or the objects it handles.
\subsubsection{Standards}
There are no applicable standards.
\subsubsection{Priority}
Critical

\subsection{Positioning with Offset}
\subsubsection{Description}
The RV8-CRL robot must achieve a maximum positioning offset of no more than 1 millimeter (0.039 inches) from the intended target position when performing precision tasks. This requirement ensures that the robot can accurately position objects and perform tasks with a high degree of precision, critical for applications such as assembly, pick-and-place operations, and quality control.
\subsubsection{Source}
This requirement is sourced by industrial robot standard.
\subsubsection{Constraints}
It may be challenging to measure performance accuracy.
\subsubsection{Standards}
ISO 9283 - This standard defines a set of tests for measuring the repeatability, absolute accuracy, and path accuracy of industrial robots
\subsubsection{Priority}
High

\subsection{Positioning with Offset}
\subsubsection{Description}
The RV8-CRL robot must achieve a maximum positioning offset of no more than 1 millimeter (0.039 inches) from the intended target position when performing precision tasks. This requirement ensures that the robot can accurately position objects and perform tasks with a high degree of precision, critical for applications such as assembly, pick-and-place operations, and quality control.
\subsubsection{Source}
This requirement is sourced by industrial robot standard.
\subsubsection{Constraints}
It may be challenging to measure performance accuracy.
\subsubsection{Standards}
ISO 9283 - This standard defines a set of tests for measuring the repeatability, absolute accuracy, and path accuracy of industrial robots
\subsubsection{Priority}
High

\subsection{Box Weight}
\subsubsection{Description}
Boxes used in the demonstration of the RV8-CRL must not exceed a weight of 8 kg (17.637 lbs). This weight limitation is imposed to safeguard the safety of robot operation and to ensure the reliability of its performance. Exceeding this weight may impede structural integrity.
\subsubsection{Source}
This requirement is sourced by industrial robot standard and Mitsubishi.
\subsubsection{Constraints}
This weight limitation restricts the amount of industrial applications that the program can be used in.
\subsubsection{Standards}
ISO 10218-1 and ISO 10218-2:2011 - These international standards provide guidelines for the safety of industrial robots. They include specifications related to the maximum payload and handling of objects by industrial robots.
\subsubsection{Priority}
High