The RV-8CRL robot arm is an industrial robot that can potentially harm those around those with electric shocks and serious injuries due to the movement of robot and wiring of robot. We shall include procedures that will make it more safer to be around the robot, and prevent any electrical hazard.

\subsection{Laboratory equipment lockout/tagout (LOTO) procedures}
\subsubsection{Description}
Any fabrication equipment provided used in the development of the project shall be used in accordance with OSHA standard LOTO procedures. Locks and tags are installed on all equipment items that present use hazards, and ONLY the course instructor or designated teaching assistants may remove a lock. All locks will be immediately replaced once the equipment is no longer in use.
\subsubsection{Source}
CSE Senior Design laboratory policy
\subsubsection{Constraints}
Equipment usage, due to lock removal policies, will be limited to availability of the course instructor and designed teaching assistants.
\subsubsection{Standards}
Occupational Safety and Health Standards 1910.147 - The control of hazardous energy (lockout/tagout).
\subsubsection{Priority}
Critical

\subsection{National Electric Code (NEC) wiring compliance}
\subsubsection{Description}
Any electrical wiring must be completed in compliance with all requirements specified in the National Electric Code. This includes wire runs, insulation, grounding, enclosures, over-current protection, and all other specifications.
\subsubsection{Source}
CSE Senior Design laboratory policy
\subsubsection{Constraints}
High voltage power sources, as defined in NFPA 70, will be avoided as much as possible in order to minimize potential hazards.
\subsubsection{Standards}
NFPA 70
\subsubsection{Priority}
Critical

\subsection{RIA robotic manipulator safety standards}
\subsubsection{Description}
Robotic manipulators, if used, will either housed in a compliant lockout cell with all required safety interlocks, or certified as a "collaborative" unit from the manufacturer.
\subsubsection{Source}
CSE Senior Design laboratory policy
\subsubsection{Constraints}
Collaborative robotic manipulators will be preferred over non-collaborative units in order to minimize potential hazards. Sourcing and use of any required safety interlock mechanisms will be the responsibility of the engineering team.
\subsubsection{Standards}
ANSI/RIA R15.06-2012 American National Standard for Industrial Robots and Robot Systems, RIA TR15.606-2016 Collaborative Robots
\subsubsection{Priority}
Critical

\subsection{Emergency Stop}
\subsubsection{Description}
Robotic arm manipulator shall stop once any of the exterior, interior, and controller emergency stops are pressed.
\subsubsection{Source}
CSE Senior Design laboratory policy
\subsubsection{Constraints}
Only 2 emergency stops will be provided that are located outside the controller. Additionally, only those allowed to enter the work cell, instructor and the engineering team, shall have access to the interior and controller emergency stop.
\subsubsection{Standards}
RV-8CRL Standards Specification
\subsubsection{Priority}
Critical

\subsection{Inductive Switch}
\subsubsection{Description}
Since the linear rail has bounds, inductive switches shall be used to prevent the robot arm from going beyond a certain limits. These inductive swtiches will be added on both ends of the linear rail.
\subsubsection{Source}
CSE Senior Design RV8 Engineering Team
\subsubsection{Constraints}
The inductive switches will only work if they are programmed into stopping the robot arm. Otherwise, they won't prevent the robot arm from traveling beyond those limits.
\subsubsection{Standards}
There are no applicable standards.
\subsubsection{Priority}
Critical
